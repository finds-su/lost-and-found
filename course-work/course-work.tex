% !TeX program  = xelatex
% !TeX encoding = UTF-8
% !TeX root     = course-work.tex
\documentclass{mirea}

% !TeX program  = xelatex
% !TeX encoding = UTF-8
% !TeX root     = course-work.tex

\usepackage{hyperref}
\hypersetup{pdftitle={Курсовая работа на тему Веб-сервис Бюро находок}, pdfauthor={В. С. Верхотуров}}

\usepackage{comment}

\usepackage{graphicx}

\usepackage{listings}
\usepackage{xcolor}
\lstset{basicstyle=\footnotesize, breaklines=true, numbers=left, captionpos=t, showstringspaces=false, commentstyle=\color{teal}, stringstyle=\color{red}, keywordstyle=\color{violet}}  % Настройки, применяемые ко всем листингам

% Создание введения или заключения
\newcommand{\supersection}[1]{
	\section*{#1}
	\phantomsection
	\addcontentsline{toc}{section}{#1}
}

\usepackage{caption}
\captionsetup[lstlisting]{justification=raggedright, singlelinecheck=false}

\begin{document}
	
\addtocounter{page}{2}

% Содержание
\tableofcontents


\supersection{Введение}

Современный мир стал свидетелем стремительного развития 
информационных технологий, которые проникают 
во все сферы нашей жизни, включая поиск и нахождение утерянных вещей. 
В ситуации, когда мы потеряли что-то ценное или важное для нас, 
возникает огромная необходимость в эффективном и 
удобном способе поиска и возврата утраченных предметов. 
Веб-сервис Бюро находок является одним из инновационных решений этой проблемы.

Целью данной курсовой работы является разработка 
и анализ веб-сер\-ви\-са Бюро находок, предоставляющего 
возможность пользователям объявлять о потерянных 
и найденных предметах, а также упрощающего процесс возврата 
утерянных вещей и связи между их владельцами и нашим сервисом.

Актуальность данного исследования обусловлена не только повседневными 
ситуациями потери вещей, но и ростом числа людей, пользующихся интернетом 
и смартфонами. Веб-сервис Бюро находок предлагает новый подход 
к организации процесса поиска и возврата утерянных предметов, 
обеспечивая удобство и оперативность взаимодействия между пользователями 
и нашим сервисом.

В аналитическом разделе будет проведен обзор существующих веб-сер\-ви\-сов 
и приложений, а также проанализированы их преимущества и недостатки. 
Специальный раздел посвящен разработке концепции Бюро находок, 
включая функциональные требования и особенности реализации. 
Технологический раздел описывает выбранные технологии и инструменты 
для разработки веб-сервиса. В экономическом разделе будет проведен расчет 
затрат на разработку и поддержку Бюро находок, а также оценена 
его экономическая эффективность. В заключении будут подведены итоги работы 
и сделаны выводы о значимости и перспективах развития веб-сервиса Бюро находок.

Для написания данной курсовой работы будут использованы различные 
источники информации, включая научные статьи, публикации, книги и данные из сети Интернет. 
Все использованные источники будут тщательно приведены 
в списке использованных литературных источников в конце работы.

Цель данного исследования заключается в создании эффективного веб-сервиса Бюро находок, 
который поможет людям быстро и надежно находить утерянные вещи 
и обеспечит удобство взаимодействия с нашим сервисом. 
В дальнейшем этот веб-сервис может стать платформой 
для реализации дополнительных функций и услуг, связанных 
с восстановлением утерянных вещей и повышением безопасности собственности.


\section{Аналитический раздел}


\subsection{Обзор существующих веб-сервисов и приложений для поиска и возврата утерянных вещей}

В настоящем разделе будет проведен обзор существующих веб-сервисов 
и приложений, которые предлагают функциональность поиска 
и возврата утерянных вещей. Данный обзор позволит выявить 
основные преимущества и недостатки этих сервисов, 
а также определить потенциальные возможности для улучшения их функциональности.

<<Lost and Found>> --- это один из наиболее популярных веб-сервисов, 
предоставляющих возможность объявлять о потерянных и найденных предметах. 
Сервис имеет простой и интуитивно понятный интерфейс, 
позволяющий пользователям быстро разместить информацию 
о потерянных вещах и связаться с владельцами найденных предметов. 
Однако, отсутствие системы уведомлений и неэффективное сопоставление 
объявлений ограничивают его функциональность.

<<Find My Stuff>> --- это мобильное приложение, разработанное 
для операционных систем iOS и Android. Оно предлагает функцию отслеживания 
утерянных предметов через GPS-модуль смартфона. Пользователи могут отмечать 
свои вещи на карте и получать уведомления, когда они находятся рядом с утерянным предметом. 
Однако, ограничение использования только наличием смартфона с GPS-модулем 
и низкая точность определения местоположения представляют 
существенные ограничения данного приложения.

<<Lost Property Office>> --- это веб-сервис, предоставляемый 
государственными организациями и органами правопорядка. 
Сервис позволяет пользователям сообщать о потерянных 
и найденных предметах, а также предоставляет информацию 
о процедуре возврата утерянных вещей. Однако, 
ограниченный доступ к сервису и неудобный процесс регистрации 
и подачи заявки являются значительными недостатками данного сервиса.

На основании проведенного обзора можно сделать вывод, 
что существующие веб-сервисы и приложения для поиска и возврата 
утерянных вещей имеют некоторые преимущества, но также недостатки, 
которые ограничивают их функциональность и удобство использования. 
Веб-сервис Бюро находок будет разработан с учетом этих недостатков 
и предлагать более эффективное и удобное взаимодействие 
между пользователями и сервисом.

\subsection{Анализ рынка и конкурентной среды}

TODO

\subsection{Исследование потребностей и предпочтений целевой аудитории}

TODO

\subsection{Проектирование архитектуры и функциональности Бюро находок}

TODO

\subsection{Анализ технических аспектов реализации веб-сервиса}

TODO

\subsection{Оценка рисков и меры по их снижению}

TODO

\subsection*{Вывод по разделу}

TODO

\section{Специальный раздел}

\subsection{Архитектура и функциональность}

TODO

\subsection{Механизм поиска и сопоставления объявлений}

TODO

\subsection{Механизм обратной связи и взаимодействия пользователей}

TODO

\subsection{Меры безопасности и конфиденциальности}

TODO

\subsection{Монетизация и бизнес-модель}

TODO

\subsection{Планы по развитию и масштабированию}

TODO

\subsection*{Вывод по разделу}

TODO

\section{Технологический раздел}

\subsection{Архитектура системы}

TODO

\subsection{Функциональные требования}

TODO

\subsection{Технические требования}

TODO

\subsection{Интерфейс пользователя}

TODO

\subsection{База данных}

TODO

\subsection{Безопасность}

TODO

\subsection{Тестирование и развертывание}

TODO

\subsection{Оптимизация и масштабирование}

TODO

\subsection*{Вывод по разделу}

TODO

\section{Экономический раздел}

\subsection{Планирование разработки программного продукта}

TODO

\subsection{Составление сметы затрат на разработку}

TODO

\subsubsection{Материальные затраты}

TODO

\subsubsection{Затраты на оплату труда}

TODO

\subsubsection{Амортизационные отчисления}

TODO

\subsubsection{Прочие расходы}

TODO

\supersection{Заключение}

TODO



\begin{thebibliography}{99\kern\bibindent}
	%\bibitem{bib:mybook} Моя книга.
	%\bibitem{bib:mybook2} Моя вторая книга. 
\end{thebibliography}



\appendix

\section{Схема базы банных}

\begin{lstlisting}[label=lst:factorial]
model Account {
	id                String  @id @default(cuid())
	userId            String
	type              String
	provider          String
	providerAccountId String
	refresh_token     String?
	access_token      String?
	expires_at        Int?
	token_type        String?
	scope             String?
	id_token          String?
	session_state     String?
	user              User    @relation(fields: [userId], references: [id], onDelete: Cascade)
	
	@@unique([provider, providerAccountId])
}

model Session {
	id           String   @id @default(cuid())
	sessionToken String   @unique
	userId       String
	expires      DateTime
	user         User     @relation(fields: [userId], references: [id], onDelete: Cascade)
	
	@@index([userId], type: Hash)
}

model User {
	id                String              @id @default(cuid())
	name              String?
	nickname          String              @unique
	socialNetworks    UserSocialNetwork[]
	email             String?             @unique
	emailVerified     DateTime?
	userInfo          String?             @db.VarChar(280)
	role              Role                @default(USER)
	image             String?
	isBlocked         Boolean             @default(false)
	blockReason       String?
	accounts          Account[]
	sessions          Session[]
	lostAndFoundItems LostAndFoundItem[]
	
	@@index([id], type: Hash)
	@@index([nickname], type: Hash)
}

model VerificationToken {
	identifier String
	token      String   @unique
	expires    DateTime
	
	@@unique([identifier, token])
}

model UserSocialNetwork {
	id                             String                           @id @default(cuid())
	socialNetwork                  SocialNetwork
	link                           String
	userId                         String
	user                           User                             @relation(fields: [userId], references: [id], onDelete: Cascade)
	lostAndFoundItemSocialNetworks LostAndFoundItemSocialNetworks[]
	
	@@unique([userId, socialNetwork])
	@@index([socialNetwork, userId])
}

enum Role {
	USER
	MODERATOR
	ADMIN
}

model LostAndFoundItem {
	id             String                           @id @default(cuid())
	name           String                           @db.VarChar(100)
	description    String                           @default("") @db.VarChar(512)
	campus         Campus
	reason         PostItemReason
	status         LostAndFoundItemStatus           @default(ACTIVE)
	images         String[]
	userId         String
	user           User                             @relation(fields: [userId], references: [id], onDelete: Cascade)
	socialNetworks LostAndFoundItemSocialNetworks[]
	created        DateTime                         @default(now())
	expires        DateTime                         @default(dbgenerated("NOW() + interval '1 week'"))
	
	@@index([id], type: Hash)
}

enum LostAndFoundItemStatus {
	ACTIVE
	EXPIRED
	BLOCKED
}

model LostAndFoundItemSocialNetworks {
	id                  String            @id @default(cuid())
	lostAndFoundItemId  String
	lostAndFoundItem    LostAndFoundItem  @relation(fields: [lostAndFoundItemId], references: [id], onDelete: Cascade)
	userSocialNetworkId String
	userSocialNetwork   UserSocialNetwork @relation(fields: [userSocialNetworkId], references: [id], onDelete: Cascade)
	
	@@unique([lostAndFoundItemId, userSocialNetworkId])
}

enum PostItemReason {
	LOST
	FOUND
}

enum Campus {
	V78
	S20
	V86
	MP1
	SG22
	SHP23
	U7
}

enum SocialNetwork {
	TELEGRAM
	VK
}
\end{lstlisting}
	
\end{document}