\section{Аналитический раздел}
\label{sec:analytics}

\subsection{Статистика потерянных и найденных вещей}

Для подтверждения актуальности и важности разрабатываемой системы, необходимо провести исследование рынка и определить основные проблемы и потребности пользователей. Одним из способов сбора информации является проведение опроса среди пользователей.

Одним из основных факторов, определяющих актуальность разрабатываемой системы является статистика потерянных и найденных вещей. Необходимо определить количество потерянных вещей в месяц, год и за весь период работы системы. Это поможет оценить нагрузку на систему и определить ее производительность.

Статистика, взятая с сайта столнаходок.рф~\cite{bib:stol_nahodok}, утверждает, что только 20~\% пользователей их сайта смогли установить и вернуть вещи. Также на рисунках \ref{fig:chart2023} и \ref{fig:chart2022} представлена гистограмма количества созданных объявлений за 2022 и 2023 года.

\begin{figure}[htb]
	\centering
	\includegraphics[width=.6\textwidth]{images/chart2023}
	\parskip=6pt
	\caption{Востребованность системы столнаходок.рф в 2023 году}
	\label{fig:chart2023}
\end{figure}

\begin{figure}[htb]
	\centering
	\includegraphics[width=.6\textwidth]{images/chart2022}
	\parskip=6pt
	\caption{Востребованность системы столнаходок.рф в 2022 году}
	\label{fig:chart2022}
\end{figure}

\subsection{Типы существующих решений для поиска и возврата утерянных вещей}

Существует несколько типов существующих решений для поиска и возврата утерянных вещей. Ниже приведены некоторые из них:
\begin{enumerate}
	\item Веб-сайты и мобильные приложения: <<Бюро находок>>~\cite{bib:stol_nahodok,bib:pona}. Эти сервисы предоставляют платформу, где люди могут регистрировать утерянные вещи и искать их владельцев. Пользователям предлагается создать объявления о найденных или потерянных вещах и связаться друг с другом, чтобы вернуть вещи. Некоторые сервисы предлагают добавить фотографии или описание вещей, чтобы облегчить поиск. 
	
	\item Технология RFID (Radio Frequency Identification) позволяет прикреплять RFID-метки к ценным объектам и определить владельца с помощью специальных считывателей~\cite{bib:investopedia_rfid,bib:airtag}. Это возможно благодаря использованию радиоволн, которые позволяют быстро определять местоположение потерянных вещей с помощью дополнительного программного обеспечения. Одним из наиболее распространенных применений технологии RFID является микрочипирование домашних животных или чипов для домашних животных. Эти микрочипы имплантируются ветеринарами и содержат информацию, касающуюся домашних животных, включая их имя, медицинские записи и контактную информацию их владельцев. Если домашнее животное пропадает и его отправляют в спасательную службу или в приют, работник приюта сканирует животное на наличие микрочипа. Если у домашнего животного есть микрочип, работнику приюта достаточно одного телефонного звонка или поиска в Интернете, чтобы связаться с владельцами домашнего животного. Считается, что чипы для домашних животных более надежны, чем ошейники, которые можно упасть или снять.
	
	\item GPS-трекеры --- это устройства с встроенным GPS-модулем. Они могут быть прикреплены практически к любому объекту, после чего его местоположение определяется через смартфон или компьютер по сети Интернет. При использовании приложения на смартфоне пользователь может получать уведомления о передвижении объекта и быстро определять его текущее местоположение.
	
	\item Автоматизированные системы поиска утерянных предметов: Некоторые организации, например, аэропорты и железнодорожные станции, используют системы обнаружения утерянных предметов. В этих системах используются технологии, такие как видеонаблюдение, детекторы движения и распознавание образов для отслеживания и возвращения потерянных предметов их владельцам.
\end{enumerate}

Каждый из этих типов решений имеет свои преимущества и недостатки. Некоторые из них могут быть более подходящими для конкретных ситуаций, например, GPS-трекеры могут быть полезными при поиске утерянных вещей на открытой местности, в то время как RFID-метки могут быть более подходящими для использования внутри помещений. Веб-сайты и приложения <<Бюро находок>> предоставляют более универсальное решение, которое может быть использовано в различных ситуациях.

\subsection{Анализ существующих систем для поиска и возврата утерянных вещей}

В настоящем разделе будет проведен обзор существующих сервисов и приложений, которые предлагают функциональность поиска и возврата утерянных вещей. Данный обзор позволит выявить основные преимущества и недостатки этих сервисов, а также определить потенциальные возможности для улучшения их функциональности.

<<столнаходок.рф>>~\cite{bib:stol_nahodok} --- это один из наиболее популярных веб-сервисов, предоставляющих возможность объявлять о потерянных и найденных предметах. Сервис имеет простой и интуитивно понятный интерфейс, позволяющий пользователям быстро разместить информацию о потерянных вещах и связаться с владельцами найденных предметов, примеры пользовательского интерфейса представлены на рис.~\ref{fig:stolNahodok1}, \ref{fig:stolNahodok2}. Однако, отсутствие системы уведомлений и неудобное сопоставление объявлений ограничивают его функциональность.

\begin{figure}[htb]
	\centering
	\includegraphics[width=.95\textwidth]{images/stolNahodok1}
	\parskip=6pt
	\caption{Скриншот системы <<столнаходок.рф>>}
	\label{fig:stolNahodok1}
\end{figure}

\begin{figure}[htb]
	\centering
	\includegraphics[width=.95\textwidth]{images/stolNahodok2}
	\parskip=6pt
	\caption{Скриншот системы <<столнаходок.рф>>}
	\label{fig:stolNahodok2}
\end{figure}

<<Стол находок>>~\cite{bib:stol_nahodok} --- это мобильное приложение, разработанное для операционных систем iOS и Android. Оно предлагает функцию отслеживания утерянных предметов через GPS-модуль смартфона, представлено на рис.~\ref{fig:findMyStuff1}, \ref{fig:findMyStuff2}. Пользователи могут отмечать свои вещи на карте и получать уведомления, когда они находятся рядом с утерянным предметом. Однако, ограничение использования только наличием смартфона с GPS-модулем и низкая точность определения местоположения представляют существенные ограничения данного приложения.

\begin{figure}[htb]
	\centering
	\includegraphics[height=.4\textheight]{images/findMyStuff1.png}
	\parskip=6pt
	\caption{Скриншот системы <<Find My Stuff>>}
	\label{fig:findMyStuff1}
\end{figure}

\begin{figure}[htb]
	\centering
	\includegraphics[height=.4\textheight]{images/findMyStuff2.png}
	\parskip=6pt
	\caption{Скриншот системы <<Find My Stuff>>}
	\label{fig:findMyStuff2}
\end{figure}

<<Lost Property Office>>~\cite{bib:parliament_lost_and_found} --- это веб-сервис, предоставляемый государственными организациями и органами правопорядка, см. рис.~\ref{fig:lostPropertyOffice}. Сервис позволяет пользователям сообщать о потерянных и найденных предметах, а также предоставляет информацию о процедуре возврата утерянных вещей. Однако, ограниченный доступ к сервису и неудобный процесс регистрации и подачи заявки являются значительными недостатками данного сервиса.

\begin{figure}[htb]
	\centering
	\includegraphics[width=.95\textwidth]{images/lostPropertyOffice}
	\parskip=6pt
	\caption{Скриншот системы <<Lost Property Office>>}
	\label{fig:lostPropertyOffice}
\end{figure}

На основании проведенного обзора можно сделать вывод, что существующие веб-сервисы и приложения для поиска и возврата утерянных вещей имеют некоторые преимущества, но также недостатки, которые ограничивают их функциональность и удобство использования. Веб-сервис Бюро находок будет разработан с учетом этих недостатков и предлагать более удобное взаимодействие между пользователями и сервисом.

Ниже приведена сравнительная таблица~\ref{tab:analogs_comparison} основных характеристик и функций приведенных выше аналогов:
\begin{table}[htb]
	\caption{Сравнительная таблица аналогов}
	\centering
	
	\tolerance=1
	\emergencystretch=10pt
	\hyphenpenalty=1
	\exhyphenpenalty=1
	\small
	\begin{tabular}{ |p{2cm}|p{3cm}|p{2cm}|p{2cm}|p{3cm}|p{2cm}| } 
		\hline
		Сервис~/ Приложение & Интерфейс и удобство использования & Опове\-ще\-ния & Точность определения местоположения & Удобство регистрации и подача заявки & Доступ\-ность \\ \hline
		
		стол\-на\-ходок.рф & Простой и интуитивно понятный интерфейс & Отсут\-ству\-ют & Не\-оп\-ре\-де\-ле\-но & Простой процесс регистрации & Широкий доступ \\ \hline
		
		Find My Stuff & Простой и интуитивно понятный интерфейс & Опо\-ве\-ще\-ния через уведомления & Низкая точность & Простой процесс регистрации & Доступен только на смартфонах с GPS \\ \hline
		
		Lost Property Office & Неудобный процесс регистрации и подачи заявки & Отсут\-ству\-ют & Не\-оп\-ре\-де\-ле\-но & Неудобный процесс регистрации и подачи заявки & Огра\-ни\-чен\-ный доступ \\ \hline
	\end{tabular}
	\label{tab:analogs_comparison}
\end{table}

\subsection*{Вывод по разделу}

В аналитической части работы проведен детальный анализ существующих веб-ресурсов и приложений, предназначенных для поиска и возвращения утерянных вещей. Были изучены и проанализированы их функциональность, характеристики, преимущества и ограничения.

Одним из наиболее популярных и востребованных решений в данной сфере являются веб-сервисы и приложения "Бюро находок". Они предоставляют пользователям платформу для регистрации утерянных вещей и связи с их владельцами, что упрощает процесс поиска и возвращения потерянных предметов.
