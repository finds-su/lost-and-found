% !TeX program  = xelatex
% !TeX encoding = UTF-8
% !TeX root     = article1.tex
\documentclass{mirea-article}

% !TeX program  = xelatex
% !TeX encoding = UTF-8
% !TeX root     = course-work.tex

\usepackage{hyperref}
\hypersetup{pdftitle={Курсовая работа на тему Веб-сервис Бюро находок}, pdfauthor={В. С. Верхотуров}}

\usepackage{comment}

\usepackage{graphicx}

\usepackage{listings}
\usepackage{xcolor}
\lstset{basicstyle=\footnotesize, breaklines=true, numbers=left, captionpos=t, showstringspaces=false, commentstyle=\color{teal}, stringstyle=\color{red}, keywordstyle=\color{violet}}  % Настройки, применяемые ко всем листингам

% Создание введения или заключения
\newcommand{\supersection}[1]{
	\section*{#1}
	\phantomsection
	\addcontentsline{toc}{section}{#1}
}

\usepackage{caption}
\captionsetup[lstlisting]{justification=raggedright, singlelinecheck=false}
\usepackage{multicol}
\usepackage{ragged2e}

\newenvironment{Figure}
{\par\medskip\noindent\minipage{\linewidth}}
{\endminipage\par\medskip}

\begin{document}
	
	\textbf{УДК 004.658.2}
	
	\begin{FlushRight}
		\textbf{Валерий Сергеевич Верхотуров}\footnote{Студент специальности «Информационные системы и технологии» ИКБ РТУ МИРЭА valery.verkhoturov1505@gmail.com} 

		\textbf{Игорь Денисович Котилевец}\footnote{Старший преподаватель кафедры ИКБ РТУ МИРЭА}
	\end{FlushRight}
	
	\supersection{Сравнительный анализ существующих цифровых решений в сфере поиска потерянных вещей.} 
	
	Аннотация. \textit{Статья представляет обзор существующих подходов и решений поиска потерянных вещей.}
	
	Ключевые слова: \textit{бюро находок, веб-сайт, RFID, GPS.}
	
	\supersection{A comparative analysis of existing digital solutions in the field of lost items retrieval}
	
	Annotation. \textit{This article provides an overview of existing approaches and solutions for finding lost items. }
	
	Keywords: \textit{lost and found, websit, RFID, GPS.}
	
	\begin{multicols}{2}
		
		
		\subsection*{Введение}
		\label{sec:introduction}
		
		Поиск утерянных вещей является актуальной проблемой, которая возникает при различных обстоятельствах. Эта проблема может возникнуть в результате потери ключей, документов, мобильных телефонов, кошельков или других ценных или важных вещей~\cite{bib:m24_losts_article,bib:usinsk_losts_article}. В связи с этим существует необходимость разработки системы, которая поможет людям вернуть утерянные вещи.
		
		Целью данной работы является проведение анализа существующих систем поиска утерянных вещей и выделение их преимуществ и недостатков.
		
		\subsection*{Статистика потерянных и найденных вещей}
		
		Для подтверждения актуальности и важности сравниваемых систем, необходимо провести исследование рынка и определить основные проблемы и потребности пользователей. Одним из способов сбора информации является проведение опроса среди пользователей.
		
		Одним из основных факторов, определяющих актуальность разрабатываемой системы является статистика потерянных и найденных вещей. Необходимо определить количество потерянных вещей в месяц, год и за весь период работы системы. Это поможет оценить нагрузку на систему и определить ее производительность.
		
		Статистика, взятая с сайта столнаходок.рф~\cite{bib:stol_nahodok}, утверждает, что только 20~\% пользователей их сайта смогли установить и вернуть вещи. Также на рисунках \ref{fig:chart2023} и \ref{fig:chart2022} представлена гистограмма количества созданных объявлений за 2022 и 2023 года.
		
		\begin{Figure}
			\centering
			\includegraphics[width=\textwidth]{../images/chart2023}
			\captionof{figure}{Востребованность системы столнаходок.рф в 2023 году}
			\label{fig:chart2023}
		\end{Figure}
	
		\begin{Figure}
			\centering
			\includegraphics[width=\textwidth]{../images/chart2022}
			\captionof{figure}{Востребованность системы столнаходок.рф в 2022 году}
			\label{fig:chart2022}
		\end{Figure}
		
		\subsection*{Типы существующих решений для поиска и возврата утерянных вещей}
		
		Существует несколько типов существующих решений для поиска и возврата утерянных вещей. Ниже приведены некоторые из них:
		\begin{enumerate}
			\item Веб-сайты и мобильные приложения: <<Бюро находок>>~\cite{bib:stol_nahodok,bib:pona}. Эти сервисы предоставляют платформу, где люди могут регистрировать утерянные вещи и искать их владельцев. Пользователям предлагается создать объявления о найденных или потерянных вещах и связаться друг с другом, чтобы вернуть вещи. Некоторые сервисы предлагают добавить фотографии или описание вещей, чтобы облегчить поиск. 
			
			\item Технология RFID (Radio Frequency Identification) позволяет прикреплять RFID-метки к ценным объектам и определить владельца с помощью специальных считывателей~\cite{bib:investopedia_rfid,bib:airtag}. Это возможно благодаря использованию радиоволн, которые позволяют быстро определять местоположение потерянных вещей с помощью дополнительного программного обеспечения. Одним из наиболее распространенных применений технологии RFID является микрочипирование домашних животных или чипов для домашних животных. Эти микрочипы имплантируются ветеринарами и содержат информацию, касающуюся домашних животных, включая их имя, медицинские записи и контактную информацию их владельцев. Если домашнее животное пропадает и его отправляют в спасательную службу или в приют, работник приюта сканирует животное на наличие микрочипа. Если у домашнего животного есть микрочип, работнику приюта достаточно одного телефонного звонка или поиска в Интернете, чтобы связаться с владельцами домашнего животного. Считается, что чипы для домашних животных более надежны, чем ошейники, которые можно упасть или снять.
			
			\item GPS-трекеры --- это устройства с встроенным GPS-модулем. Они могут быть прикреплены практически к любому объекту, после чего его местоположение определяется через смартфон или компьютер по сети Интернет. При использовании приложения на смартфоне пользователь может получать уведомления о передвижении объекта и быстро определять его текущее местоположение.
			
			\item Автоматизированные системы поиска утерянных предметов: Некоторые организации, например, аэропорты и железнодорожные станции, используют системы обнаружения утерянных предметов. В этих системах используются технологии, такие как видеонаблюдение, детекторы движения и распознавание образов для отслеживания и возвращения потерянных предметов их владельцам.
		\end{enumerate}
		
		Каждый из этих типов решений имеет свои преимущества и недостатки. Некоторые из них могут быть более подходящими для конкретных ситуаций, например, GPS-трекеры могут быть полезными при поиске утерянных вещей на открытой местности, в то время как RFID-метки могут быть более подходящими для использования внутри помещений. Веб-сайты и приложения <<Бюро находок>> предоставляют более универсальное решение, которое может быть использовано в различных ситуациях.
		
		\subsection*{Анализ существующих систем для поиска и возврата утерянных вещей}
		
		В настоящем разделе будет проведен обзор существующих сервисов и приложений, которые предлагают функциональность поиска и возврата утерянных вещей. Данный обзор позволит выявить основные преимущества и недостатки этих сервисов, а также определить потенциальные возможности для улучшения их функциональности.
		
		<<столнаходок.рф>>~\cite{bib:stol_nahodok} --- это один из наиболее популярных веб-сервисов, предоставляющих возможность объявлять о потерянных и найденных предметах. Сервис имеет простой и интуитивно понятный интерфейс, позволяющий пользователям быстро разместить информацию о потерянных вещах и связаться с владельцами найденных предметов, примеры пользовательского интерфейса представлены на рис.~\ref{fig:stolNahodok1}, \ref{fig:stolNahodok2}. Однако, отсутствие системы уведомлений и неудобное сопоставление объявлений ограничивают его функциональность.
		
		\begin{Figure}
			\centering
			\includegraphics[width=\textwidth]{../images/stolNahodok1}
			\captionof{figure}{Скриншот системы <<столнаходок.рф>>}
			\label{fig:stolNahodok1}
		\end{Figure}
		
		<<Find My Stuff>>~\cite{bib:find_my_stuff} --- это мобильное приложение, разработанное для операционных систем iOS и Android. Оно предлагает функцию отслеживания утерянных предметов через GPS-модуль смартфона, представлено на рис.~\ref{fig:findMyStuff1}. Пользователи могут отмечать свои вещи на карте и получать уведомления, когда они находятся рядом с утерянным предметом. Однако, ограничение использования только наличием смартфона с GPS-модулем и низкая точность определения местоположения представляют существенные ограничения данного приложения.
		
		\begin{Figure}
			\centering
			\includegraphics[width=.5\textwidth]{../images/findMyStuff1.png}
			\captionof{figure}{Скриншот системы <<Find My Stuff>>}
			\label{fig:findMyStuff1}
		\end{Figure}
		
		<<Lost Property Office>>~\cite{bib:parliament_lost_and_found} --- это веб-сервис, предоставляемый государственными организациями и органами правопорядка, см. рис.~\ref{fig:lostPropertyOffice}. Сервис позволяет пользователям сообщать о потерянных и найденных предметах, а также предоставляет информацию о процедуре возврата утерянных вещей. Однако, ограниченный доступ к сервису и неудобный процесс регистрации и подачи заявки являются значительными недостатками данного сервиса.
		
		\begin{Figure}
			\centering
			\includegraphics[width=.95\textwidth]{../images/lostPropertyOffice}
			\captionof{figure}{Скриншот системы <<Lost Property Office>>}
			\label{fig:lostPropertyOffice}
		\end{Figure}
		
		На основании проведенного обзора можно сделать вывод, что существующие веб-сервисы и приложения для поиска и возврата утерянных вещей имеют некоторые преимущества, но также недостатки, которые ограничивают их функциональность и удобство использования. Веб-сервис Бюро находок будет разработан с учетом этих недостатков и предлагать более удобное взаимодействие между пользователями и сервисом.
		
		\subsection*{Вывод}
		
		В статье проведен детальный анализ существующих веб-ресурсов и приложений, предназначенных для поиска и возвращения утерянных вещей. Были изучены и проанализированы их функциональность, характеристики, преимущества и ограничения.
		
		Одним из наиболее популярных и востребованных решений в данной сфере являются веб-сервисы и приложения "Бюро находок". Они предоставляют пользователям платформу для регистрации утерянных вещей и связи с их владельцами, что упрощает процесс поиска и возвращения потерянных предметов.
	\end{multicols}
	
	\begin{thebibliography}{99\kern\bibindent}
		\bibitem{bib:m24_losts_article} МОСКВА 24 Что теряют москвичи // www.m24.ru: Новости Москвы, репортажи и интервью об основных событиях города URL: \url{https://www.m24.ru/news/gorod/28112019/98853} (дата обращения: 01.09.2023).
		
		\bibitem{bib:usinsk_losts_article} Усинск Онлайн Какие вещи чаще всего теряют россияне // usinsk.online URL: \url{https://usinsk.online/news/kakie-veshhi-chashhe-vsego-teryayut-rossiyane/#:~:text=Чаще%20всего%20россияне%20теряют%3A%20кошельки,1%20процент)%2C%20пишет%20РГ.} (дата обращения: 01.09.2023).
		
		\bibitem{bib:about1} Bataineh, Emad, Bilal Bataineh, and Shama Al Kindi. "Design, development and usability evaluation of an online web-based lost and found system." International Journal of Digital Information and Wireless Communications 5.2 (2015): 75-82. % https://citeseerx.ist.psu.edu/document?repid=rep1&type=pdf&doi=c757d2b9e8b8ba4235342217fab983e2f89f6bd0
		
		\bibitem{bib:about2} Tan, Siok Yee, and Cia Rui Chong. "AN EFFECTIVE LOST AND FOUND SYSTEM IN UNIVERSITY CAMPUS." Management 8.32: 99-112. % http://www.jistm.com/PDF/JISTM-2023-32-09-07.pdf
		
		
		\bibitem{bib:stol_nahodok} Бюро находок // столнаходок.рф: информационно-поисковый портал РФ URL: \url{http://nahodok.ru/} (дата обращения: 01.09.2023).
		
		\bibitem{bib:pona} Потерял Нашел // pona1.ru: бюро находок Пона.рф. Удобный поиск по объявлениям, большая база потерянных вещей и животных URL: \url{https://pona1.ru/sochi} (дата обращения: 01.09.2023).
		
		\bibitem{bib:investopedia_rfid} Investopedia // investopedia.com: Radio Frequency Identification (RFID): What It Is, How It Works URL: \url{https://www.investopedia.com/terms/r/radio-frequency-identification-rfid.asp#:~:text=Radio%20Frequency%20Identification%20(RFID)%20is,checked%20out%20of%20a%20library.} (дата обращения: 01.09.2023).
		
		\bibitem{bib:investopedia_rfid} Investopedia // investopedia.com: Radio Frequency Identification (RFID): What It Is, How It Works URL: \url{https://www.investopedia.com/terms/r/radio-frequency-identification-rfid.asp#:~:text=Radio%20Frequency%20Identification%20(RFID)%20is,checked%20out%20of%20a%20library.} (дата обращения: 01.09.2023).
		
		\bibitem{bib:airtag} AirTag // apple.com: магазин Apple URL: \url{https://www.apple.com/airtag/} (дата обращения: 01.09.2023).
		
		\bibitem{bib:parliament_lost_and_found} Lost Property Office // parliament.uk: веб приложение URL: \url{https://www.parliament.uk/visiting/access/facilities/lost-property/} (дата обращения: 01.09.2023).
	\end{thebibliography}
	
\end{document}
